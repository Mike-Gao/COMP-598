\documentclass[a4paper, 11pt]{article}
\usepackage{comment} % enables the use of multi-line comments (\ifx \fi)  
\usepackage{fullpage} % changes the margin
\usepackage{mathtools} %allows us to write complex equations
\usepackage{graphicx} %allows us to add pictures
\usepackage{amsmath} %allows us to add Greek letters and equations
\usepackage{amssymb}
\usepackage{float} %formatting pictures
\usepackage{afterpage}
\usepackage{tikz}
\usepackage{minted}
\usetikzlibrary{automata, positioning, arrows}
\tikzset{->,>=stealth',node distance=3cm}

\begin{document}
\noindent
\large\textbf{Homework 4} \hfill \\
\large{Mike Gao} \\
\normalsize 260915701 \\
Prof. Prakash Panangaden \hfill 


\section*{Question 1}
This is decidable for a given input. $T=(Q,\Sigma,\Gamma,\delta,q0)$ be the turing machine in question, and $s$ the given input. There are only $598*|Q|*|\Gamma|^{598}$ configurations that T can be in that the machine has never use more than 598 cells. So we run the machine for $598*|Q|*|\Gamma|^{598}+1$ steps, Using the pigenhole principle, we know that if a configuration is repeated, the machine never use more than 598 cells, otherwise, the machine will use more than 598 cells.

\section*{Question 2}
We can run membership testing algorithm in $L(G_1)$ and $L(G_2)$ on every word, and such tester is sure to terminate. If we find a word both in $L(G_1)$ and $L(G_2)$ we know the statement is false. However, the case is undecidable when we are determining whether the statement is true, because $VALCOMPS_2(M,w)$ is empty only if turing machine $M$ does not accept w. Thus we have a reduction to $\lnot A_{TM}$ which is undecidable.

\section*{Question 3}
The parameters to determine the motion of the submarine is its starting location $(x,y)$, its velocity $v$ and direction. 

At each step $n$ we make a guess of those parameters. For example, if we guess direction 'up', we will get $(x,y+n*v)$. We simply try every possible combination.

We will use the idea of dovetailing. Since $(x,y)$ is an integer pair, and $v$ is a natual number, we can map it to a single natual number $n$ using a variation of cantor's paring function $f:\mathbb {Z}\times\mathbb{Z}\times\mathbb{N}\times\{0,1,2,3\}\rightarrow\mathbb{N}$. Where 0 represents up, 1 represents right, 2 represents left, 3 represents down.

\section*{Question 4}
\subsection*{4.1}
Suppose a DFA $D=(Q,\Sigma, \delta, F, q_0)$ accepts L. We define a new DFA $D'=(Q,\Sigma, \delta, F', q_0)$ which has the same start state, transition and alphabet as $D$, the only difference is the final state, where $F'=\{q\in Q| \delta(q,w) \in F\}$. This DFA recognizes $L/w$ so it is regular.

\subsection*{4.2}
Consider the language $L= N\#\Sigma^* \cup \Sigma^*\#L(G)$. We know $\Sigma^*$ is regular. Since N is context free, we have that $N\#\Sigma^*$ and $\Sigma^*\#L(G)$ are also context free. Since the union of two context free languages are context free, we get that L is also context free.

Now we have two cases:

$L(G) = \Sigma^*$, the language L is just $\Sigma^*\#\Sigma^*$ which is regular

$L(G) \neq \Sigma^*$, and assume there exist $s \in \Sigma^* \land s \notin L(G) $. Consider $L/\#s=N$. Assume N is not regular, so $L/\#s$ and hence L is not regular. So we know that L is regular if and only if $L(G) = \Sigma^*$ But this is known to be undecidable.

\section*{Question 5}
\subsection*{5.1}
$R \subseteq L$ is undecidable.

It is a well known problem that $L=\Sigma^*$ is undecidable. Since $\Sigma^*$ is a regular language, it is undecidable whether $R \subseteq L$

\subsection*{5.2}
$L \subseteq R$ is decidable.

$L \subseteq R$ iff $L\cap \overline R = nil$, So $L\cap \overline R$ is context free, and therefore $L \subseteq R$ is decidable.

\section*{Question 6}
Your favorite language, OCaml.
\begin{minted}{ocaml}
(fun s -> Printf.printf s (string_of_format s)) 
    "(fun s -> Printf.printf s (string_of_format s)) %S;;";;
\end{minted}



\end{document}
